\section{Learning outcomes}
\label{sec:Learngin-Outcomes}

\subsection{Analysis}
\label{subsec:Analysis}

The first step in creating a useful tool for the company is to define both functional and non-functional requirements, since they will shape the direction of the project from the very beginning. These requirements will be the outcome of the analysis process specified in this document.

First and foremost, the target audience and key stakeholders will be identified. Without this step, it would be difficult to narrow the focus to a specific group of people, which would make the requirements too broad and their validation much harder. Key stakeholders will mainly be used to validate the gathered requirements because they are the most interested in the project and can influence it the most. That being said, it is expected that stakeholder analysis should be completed in the first two days of Sprint 1.

Once the target audience is known, initial interviews with selected individuals will be conducted. The list of individuals for the interviews will be retrieved by contacting the key stakeholders or the HR department of the company. The aim of these interviews is to gain a better understanding of the bottlenecks and hurdles in the current process that this project is intended to address. The interviews will be structured, meaning that selected candidates will receive a list of questions in advance to ensure that all questions are fully answered. To better evaluate the responses, the interviews will be recorded (with the interviewee’s consent). After the initial interviews are analyzed, a list of requirements will be compiled and validated with the key stakeholders of the project. Interviews should be scheduled during the first week of Sprint 1 and conducted on a selected day in the second week. The following day should be reserved for gathering and reviewing the collected information.

While interviews will help identify the most obvious requirements, some may still be missed if interviewees are not fully prepared. To address this, an observation method will also be used, where two team members will observe the process that the product is meant to assist and take notes. Additional requirements identified this way will also be validated with key stakeholders. This should be done in parallel with the interview process to save time, meaning it will mirror the interview schedule.

Since there is an in-house application that has been tested but is not yet fully utilized, interviews with its developers will be conducted to clarify how the new product should integrate with the existing system. The conclusions from these interviews will mainly affect the functional requirements created earlier.

Based on the steps above, a list of functional and non-functional requirements will be created. This list will then be reviewed by the team and categorized according to the estimated size and complexity of each requirement. As a final step, the platform that will host these requirements will be chosen for better organization and progress tracking.

Finally, a paper prototype with wireframes will be created to represent the most important requirements from the requirements list. This paper prototype will serve as the final validation step, ensuring that the gathered functional and non-functional requirements are aligned with the client’s interests. 


 

\subsection{Available solutions}
\label{subsec:Available solutions}

\subsection{Proposed solution}
\label{subsec:Proposed solution}

\subsection{Advice}
\label{subsec:Advice}

